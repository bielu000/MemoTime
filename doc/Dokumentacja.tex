\documentclass[a4paper]{article}
\usepackage{polski}
\usepackage[cp1250]{inputenc}
\usepackage{url}
\usepackage{indentfirst}

\title{\bf{Menadżer zadań -- MemoTime}}
\author{{\em Biel Patryk, Janusz Anna, Oleksy Kamil}}
\date{}

\begin{document}

\begin{titlepage}
\maketitle
\thispagestyle{empty}
\bigskip
\begin{center}
Zespołowe przedsięwzięcie inżynierskie\\[2mm]

Informatyka\\[2mm]

Rok. akad. 2017/2018, sem. I\\[2mm]

Prowadzący: dr hab. Marcin Mazur
\end{center}
\end{titlepage}

\tableofcontents
\thispagestyle{empty}

\newpage

\section{Opis projektu}
\label{sec:OpisProjektu}

\subsection{Członkowie zespołu}
\label{subsec:CzlonkowieZespolu}

\begin{enumerate}
\item Anna Janusz (kierownik projektu)
\item Patryk Biel
\item Kamil Oleksy
\end{enumerate}

\subsection{Cel projektu (produkt)}
\label{subsec:CelProjektu}

Cel projektu to stworzenie aplikacji MemoTime, dzięki której możesz efektywniej organizować swój czas. 
Złożone projekty czy pojedyncze zadania, które można wprowadzić do aplikacji pozwolą na wydajne działanie. 

\subsection{Potencjalny odbiorca produktu (klient)}
\label{subsec:PotencjalnyOdbiorcaProduktu}

Ile razy zapomnieliśmy zrobić coś ważnego, czy zapomnieć o urodzinach? Dla wielu osób uporządkowana lista zadań to najlepszy i jedyny sposób na zajęcie się wszystkimi obowiązkami choć i są osoby które mają z tym problem. "MemoTime" ma za zadanie pomóc każdemu. 

\subsection{Metodyka}
\label{subsec:Metodyka}

Projekt będzie realizowany przy użyciu (zaadaptowanej do istniejących warunków) metodyki {\em Scrum}. 

\section{Wymagania użytkownika}
\label{sec:WymaganiaUzytkownika}
Struktura historyjek (User Story):\\
\textbf{Jako} \textit{kto/kiedy/gdzie}, \textbf{chcę} \textit{co}, \textbf{ponieważ}  \textit{dlaczego} \textbf{warunki satysfakcji}.

\subsection{User story 1}
\label{subsec:UserStory1}
Jako użytkownik chce mieć możliwość zarejestrowania się i zalogowania się do aplikacji za pomocą adresu e-mail lub loginu, oraz hasła, bo chce mieć zagwarantowane bezpieczeństwo i pewność że nikt nie będzie widzieć moich planów.

\subsection{User story 2}
\label{subsec:UserStory2}
Jako użytkownik chcę mieć możliwość dodawania projektów, zadań do projektów bo pozwoli mi się to skupić na jednym zadaniu i lepiej zorganizować sobie czas.

\subsection{User story 3}
\label{subsec:UserStory3}
Jako użytkownik chcę mieć możliwość modyfikowania/ usuwania zadań bo być może zmienię zdanie bądź zadanie wygaśnie a chce mieć możliwość ich edycji.

\subsection{User story 4}
\label{subsec:UserStory4}
Jako użytkownik chcę mieć możliwość zaznaczania zadań, które zostały wykonane, bo będę widział, co już zostało zrobione i nie będę zawracał sobie już tym głowy.

\subsection{User story 5}
\label{subsec:UserStory5}

Jako użytkownik chcę mieć możliwość priorytetowania zadań / etykietowania zadań, bo pozwoli mi to widzieć co jest ważniejsze do zrobienia i pomoże mi szybciej szukać zadań, które mam do wykonania (np. kontekst - w domu).

\subsection{User story 6}
\label{subsec:UserStory6}

Jako użytkownik chcę mieć możliwość dodawania podzadań do zadania, dzięki czemu będę mógł tworzyć tzw. checklisty i robić listę zakupów czy listę rzeczy do zrobienia na studia.

\subsection{User story 7}
\label{subsec:UserStory7}

Jako użytkownik chcę mieć możliwość przypominania mi o zadaniach do wykonania  przez wysyłanie dziennego zrzutu zadań na e-mail, bo czasem nie będę mieć dostępu do komputera a nie chce o niczym zapomnieć.

\subsection{User story 8}
\label{subsec:UserStory8}

Jako użytkownik chcę mieć możliwość drukowania zadań/projektów, aby np. powiesić je sobie na lodówce, aby mi przypominały o tym co jest do zrobienia.

\subsection{User story 9}
\label{subsec:UserStory9}

Jako użytkownik chcę mieć możliwość widoku kalendarza i dodatkowej funkcji dodawania specjalnych dat(np. urodziny) ze specjalną ikonką , aby wzrokowo widzieć ważne dla mnie daty jak i rozkład swoich zadań.



\section{Harmonogram}
\label{sec:Harmonogram}

\subsection{Rejestr zadań (Product Backlog)}
\label{subsec:RejestrZadan}

\begin{itemize}
\item Data rozpoczęcia: <<data>>.
\item  Data zakończenia: <<data>>.
\end{itemize}

\subsection{Sprint 1}

\begin{itemize}
\item Data rozpoczęcia: <<data>>.
\item Data zakończenia: <<data>>.
\item Scrum Master: <<imię i nazwisko>>.
\item Product Owner: <<imię i nazwisko>>.
\item Development Team: <<lista developerów>>.
\end{itemize}

\subsection{Sprint 2}

\begin{itemize}
\item Data rozpoczęcia: <<data>>.
\item  Data zakończenia: <<data>>.
\item Scrum Master: <<imię i nazwisko>>.
\item Product Owner: <<imię i nazwisko>>.
\item Development Team: <<lista developerów>>.
\end{itemize}

\subsection*{<<Tutaj dodawać kolejne Sprint'y>>}

\section{Product Backlog}

\subsection{Backlog Item 1}
\paragraph{Tytuł zadania.} <<Tytuł>>.
\paragraph{Opis zadania.} <<Opis>>.
\paragraph{Priorytet.} <<Priorytet>>.
\paragraph{Definition of Done.} <<Określić (w języku zrozumiałym dla wszystkich członków zespołu), co oznacza ukończenie danego zadania>>.

\subsection{Backlog Item 2}
\paragraph{Tytuł zadania.} <<Tytuł>>.
\paragraph{Opis zadania.} <<Opis>>.
\paragraph{Priorytet.} <<Priorytet>>.
\paragraph{Definition of Done.} <<Określić (w języku zrozumiałym dla wszystkich członków zespołu), co oznacza ukończenie danego zadania>>.

\subsection*{<<Tutaj dodawać kolejne zadania>>}

\section{Sprint 1}
\subsection{Cel} <<Określić, w jakim celu tworzony jest przyrost produktu>>.
\subsection{Sprint Planning/Backlog}

\paragraph{Tytuł zadania.} <<Tytuł>>.
\begin{itemize}
\item Estymata: <<szacowana czasochłonność (w ,,koszulkach'')>>.
\end{itemize}

\paragraph{Tytuł zadania.} <<Tytuł>>.
\begin{itemize}
\item Estymata: <<szacowana czasochłonność (w ,,koszulkach'')>>.
\end{itemize}

\paragraph{<<Tutaj dodawać kolejne zadania>>}

\subsection{Realizacja}

\paragraph{Tytuł zadania.} <<Tytuł>>.
\subparagraph{Wykonawca.} <<Wykonawca>>.
\subparagraph{Realizacja.} <<Sprawozdanie z realizacji zadania (w tym ocena zgodności z estymatą). Kod programu (środowisko \texttt{verbatim}): \begin{verbatim}
for (i=1; i<10; i++)
...
\end{verbatim}>>.

\paragraph{Tytuł zadania.} <<Tytuł>>.
\subparagraph{Wykonawca.} <<Wykonawca>>.
\subparagraph{Realizacja.} <<Sprawozdanie z realizacji zadania (w tym ocena zgodności z estymatą). Kod programu (środowisko \texttt{verbatim}): \begin{verbatim}
for (i=1; i<10; i++)
...
\end{verbatim}>>.

\paragraph{<<Tutaj dodawać kolejne zadania>>}


\subsection{Sprint Review/Demo}
<<Sprawozdanie z przeglądu Sprint'u -- czy założony cel (przyrost) został osiągnięty oraz czy wszystkie zaplanowane Backlog Item'y zostały zrealizowane? Demostracja przyrostu produktu>>.

\section{Sprint 2}

\subsection{Cel} <<Określić, w jakim celu tworzony jest przyrost produktu>>.

\subsection{Sprint Planning/Backlog}

\paragraph{Tytuł zadania.} <<Tytuł>>.
\begin{itemize}
\item Estymata: <<szacowana czasochłonność (w ,,koszulkach'')>>.
\end{itemize}

\paragraph{Tytuł zadania.} <<Tytuł>>.
\begin{itemize}
\item Estymata: <<szacowana czasochłonność (w ,,koszulkach'')>>.
\end{itemize}

\paragraph{<<Tutaj dodawać kolejne zadania>>}

\subsection{Realizacja}

\paragraph{Tytuł zadania.} <<Tytuł>>.
\subparagraph{Wykonawca.} <<Wykonawca>>.
\subparagraph{Realizacja.} <<Sprawozdanie z realizacji zadania (w tym ocena zgodności z estymatą). Kod programu (środowisko \texttt{verbatim}): \begin{verbatim}
for (i=1; i<10; i++)
...
\end{verbatim}>>.

\paragraph{Tytuł zadania.} <<Tytuł>>.
\subparagraph{Wykonawca.} <<Wykonawca>>.
\subparagraph{Realizacja.} <<Sprawozdanie z realizacji zadania (w tym ocena zgodności z estymatą). Kod programu (środowisko \texttt{verbatim}): \begin{verbatim}
for (i=1; i<10; i++)
...
\end{verbatim}>>.

\paragraph{<<Tutaj dodawać kolejne zadania>>}


\subsection{Sprint Review/Demo}
<<Sprawozdanie z przeglądu Sprint'u -- czy założony cel (przyrost) został osiągnięty oraz czy wszystkie zaplanowane Backlog Item'y zostały zrealizowane? Demostracja przyrostu produktu>>.

\section*{<<Tutaj dodawać kolejne Sprint'y>>}


\begin{thebibliography}{9}

\bibitem{Cov} S. R. Covey, {\em 7 nawyków skutecznego działania}, Rebis, Poznań, 2007.

\bibitem{Oet} Tobias Oetiker i wsp., Nie za krótkie wprowadzenie do systemu \LaTeX  \ $2_\varepsilon$, \url{ftp://ftp.gust.org.pl/TeX/info/lshort/polish/lshort2e.pdf}

\bibitem{SchSut} K. Schwaber, J. Sutherland, {\em Scrum Guide}, \url{http://www.scrumguides.org/}, 2016.

\bibitem{apr} \url{https://agilepainrelief.com/notesfromatooluser/tag/scrum-by-example}

\bibitem{us} \url{https://www.tutorialspoint.com/scrum/scrum_user_stories.htm}

\end{thebibliography}

\end{document}

% ----------------------------------------------------------------
